\documentclass[runningheads]{format/llncs}
\usepackage{preamble}

\makeatother
\title{
    On the Impossibilty of\\
    Coloring Turing Machines
}

\author{Aleksis Brezas\and Dionysis Zindros\inst{1}}

\institute{University of Athens}

\begin{document}

\maketitle

A well known result in computability theory, due to
Chaitin~\cite{chaitin}, states that whether
a given machine is \emph{the shortest} among its peers is an
uncomputable and non-enumerable property. A natural question
that arises is whether other properties of Turing Machines are
computable or not. Rice~\cite{rice} showed that this question
is undecidable for all non-trivial semantic properties of machines.
Here, we turn our attention to non-semantic properties of
machines. We show that any coloring of machines which colors
at least one representative machine for every computable problem
is non-enumerable. This formulation constitutes an interesting
generalization of both the Halting Problem~\cite{turing} and
Chaitin's Incompleteness Theorem. While the proof methods are
well known in the literature, we believe the formulation of the
problem in the form of machine coloring provides some alternative
insight into these results and can prove educational.

\section{The Machine Coloring Problem}
Halting machines can be partitioned into equivalence classes based
on the decision problem they solve.
Among all the machines that solve a particular decision problem, some machines
may be preferred in comparison to others. For example the
algorithmically \emph{most efficient} machine among its peers, or the
machine with the \emph{shortest description}~\cite{kolmogorov} seem of interest.
Each of these properties elects one or more \emph{representatives} among
all the machines that solve a particular problem. We are interested
in any property which elects \emph{at least} one representative from
the equivalence class of machines that compute the same problem.
We call such properties \emph{colorings} of machines.

\begin{definition}[Machine Coloring]
We define a machine \emph{coloring} to be any predicate evaluated
on Turing Machines with the following requirements:

\begin{enumerate}
    \item For every class of machines that compute the
          same problem, the predicate
          is \emph{true} for at least one machine in the class.

    \item The predicate is \emph{false} for a machine if there
          exists an input for which the machine does not halt.
\end{enumerate}
\end{definition}

Observe that the trivial predicates
applying on Turing Machines such as the predicate
``it is \emph{not} a Turing Machine'' which is always \emph{false}, or the
predicate ``it \emph{is} a Turing Machine'', which is always \emph{true},
are not colorings. The former does not evaluate to \emph{true}
for any class, and so it fails to satisfy the first requirement.
The latter evaluates to \emph{true} even
for non-halting machines, and so it fails to satisfy the
second requirement.

Two interesting colorings are:

\begin{enumerate}
    \item The coloring of \emph{Turing Machines that always halt}.
          This colors every machine that always halts, but
          leaves every machine that does not always halt uncolored.
    \item The coloring of the machine with the \emph{shortest description}
          among all the machines computing the same problem.
\end{enumerate}

The first coloring elects multiple representatives per class. In
fact, it colors every machine in every class of machines, if we
consider the classes of machines based on whether they decide
the same problem (it leaves machines that do not decide problems
uncolored). The second coloring
elects a finite number of representatives per class (and only one
if there is a unique shortest machine among its peers).

We now examine the question of whether such colorings are
computable, the \emph{Machine Coloring Problem}.
As we will see, not only do they fail to be computable,
but they also fail to be enumerable. Our theorem is applied on
generic colorings. Applying it to the first coloring example
above gives rise to the uncomputability
of the Halting Problem. Applying it to the second example
gives rise to Chaitin's Incompressibility Theorem.

\section{Colorings are Uncomputable}
Our result indicates that, regardless of the coloring choice,
determining the color of a machine is uncomputable and unenumerable.
Our proof follows a standard diagonal argument.

\begin{theorem}[Coloring Unenumerability]\label{thm:color}
    Any machine coloring is non-enumerable.
\end{theorem}
\begin{proof}
    Consider an arbitrary machine coloring $p$. Suppose for
    contradiction that there exists a machine $M$ which enumerates
    all the machines for which $p$ holds.

    Then we construct the following machine $\Delta$:

    \begin{quote}
        `` On input $i \in \mathbb{N}$:
        \begin{itemize}
            \item Run $M$ until you obtain its $i^\text{th}$ output, $M^*$.
            \item Run $M^*$ on input $i$ to obtain output $b$.
            \item Output $\lnot b$. ''
        \end{itemize}
    \end{quote}

    $\Delta$ is an always-terminating machine, because for every $i$, the
    enumerator $M$ will always output an $i^\text{th}$ machine.
    This is because there are infinite solvable problems, and
    there exists a colored machine for every one of them.
    Furthermore, the machine $M^*$ is always-terminating (due to the
    second requirement of colorings), and so
    the evaluation $M^*(i)$ will halt.

    Now observe that the machine $\Delta$ cannot be equivalent to any colored
    machine. This is because,
    for the $i^\text{th}$ colored machine $M^*$, we have
    $\Delta(i) \neq M^*(i)$. However, by the first requirement of colorings,
    there must exist a colored representative
    $M^*$ in the equivalence class $[\Delta]$, which is a contradiction.
    \qed
\end{proof}

\section{Relationship of Colorings to Other Problems}

Applying our theorem to two interesting colorings allows us to prove
the two well-known theorems from the theory of computation, namely
that the Halting Problem is uncomputable as well as Chaitin's Incompleteness
Theorem. We provide them as corollaries here.

\begin{corollary}[Turing's Halting Unenumerability]
    The set of Turing Machines that halt on every input is
    non-enumerable.
\end{corollary}
\begin{proof}
    Apply Theorem~\ref{thm:color}
    to the coloring which colors exactly all machines that
    always halt.
    \qed
\end{proof}

\begin{corollary}[Chaitin's Incompleteness]
    The set of Kolmogorov-minimal Turing Machines is
    non-enumerable.
\end{corollary}
\begin{proof}
    Apply Theorem~\ref{thm:color}
    to the coloring which colors exactly those machines
    that are minimal for their respective classes.
    \qed
\end{proof}

\section{Conclusion}
In this work, we considered properties of Turing Machines and
their computability. Going beyond the semantic properties of Rice's
theorem, we considered properties that have at least one
representative for each equivalence class, but no representatives
for machines that do not decide any problems. Using a diagonal
argument, we showed that such properties, which we call
machine \emph{colorings}, are not only uncomputable, but also
non-enumerable. Our theorem is a generalization of both Turing's
Halting Problem result, as well as Chaitin's Incompleteness Theorem.
We hope that this connection between the two problems provides some
insight and that this alternative way of looking at these
foundational results can be educational.

\bibliographystyle{format/splncs04}
\bibliography{biblio}

\end{document}
